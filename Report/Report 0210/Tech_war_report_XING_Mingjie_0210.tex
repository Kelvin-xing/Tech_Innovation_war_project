%				\begin{itemize}
%				\item \textbf{Time: }
%				\item \textbf{Act: }
%				\item \textbf{Response: }
%				\item	\textbf{Source: }
%				\end{itemize}


% This is a simple template for a LaTeX document using the "article" class.
% See "book", "report", "letter" for other types of document.

\documentclass[12pt]{article} % use larger type; default would be 10pt

\usepackage[utf8]{inputenc} % set input encoding (not needed with XeLaTeX)

%%% Examples of Article customizations
% These packages are optional, depending whether you want the features they provide.
% See the LaTeX Companion or other references for full information.

%%% PAGE DIMENSIONS
\usepackage{geometry} % to change the page dimensions
\geometry{a4paper} % or letterpaper (US) or a5paper or....
\usepackage{setspace}
\usepackage{parskip}
\parskip = 0.3 \baselineskip %\advance\parskip by 0pt plus 2pt% to change between paragraphs space
% \geometry{margin=2in} % for example, change the margins to 2 inches all round
% \geometry{landscape} % set up the page for landscape
%   read geometry.pdf for detailed page layout information

% \usepackage{gravarphicx} % support the \includegravarphics command and options
% \usepackage[parfill]{parskip} % Activate to begin paragraphs with an empty line rather than an indent

%%% PACKAGES
\usepackage{booktabs} % for much better looking tables
\usepackage{array} % for better arrays (eg matrices) in maths
\usepackage{paralist} % very flexible & customisable lists (eg. enumerate/itemize, etc.)
\usepackage{verbatim} % adds environment for commenting out blocks of text & for better verbatim
\usepackage{subfig} % make it possible to include more than one captioned figure/table in a single float
% These packages are all incorporated in the memoir class to one degree or another...
\usepackage[fleqn]{amsmath}
\usepackage{amssymb}
\usepackage{enumitem}
\usepackage{amsthm}
\usepackage{graphicx}
\usepackage{filecontents}
\usepackage{natbib}
\usepackage{blindtext}
\usepackage{titlesec}


%%% HEADERS & FOOTERS
\usepackage{fancyhdr} % This should be set AFTER setting up the page geometry
\pagestyle{fancy} % options: empty , plain , fancy
\renewcommand{\headrulewidth}{0pt} % customise the layout...
\lhead{}\chead{}\rhead{}
\lfoot{}\cfoot{\thepage}\rfoot{}

%%% SECTION TITLE APPEARANCE
\usepackage{sectsty}
\allsectionsfont{\rmfamily\bfseries\upshape} % (See the fntguide.pdf for font help)
% (This matches ConTeXt defaults)

%%% ToC (table of contents) APPEARANCE
\usepackage[nottoc,notlof,notlot]{tocbibind} % Put the bibliography in the ToC
\usepackage[titles,subfigure]{tocloft} % Alter the style of the Table of Contents
\renewcommand{\cftsecfont}{\rmfamily\mdseries\upshape}
\renewcommand{\cftsecpagefont}{\rmfamily\mdseries\upshape} % No bold!

\usepackage[colorlinks,citecolor=black,urlcolor=black,bookmarks=false,hypertexnames=true]{hyperref} 
%%% END Article customizations



%%% The "real" document content comes below...

\title{US Tech and Trade Policy Report}
\author{Xing Mingjie}
\date{\today} % Activate to display a given date or no date (if empty),
         % otherwise the current date is printed 

\begin{document}
\maketitle

\tableofcontents

\newpage





\section{US Tech War Timeline in the 21st Century}
	\subsection{China}
		\subsubsection{Pre-Trump Era}
			\begin{itemize}
			\item	Obama complaints at WTO in joint with EU and Japan on China's control of rare earths
				\begin{itemize}
				\item	\textbf{Time}: 14 March 2012
				\item \textbf{Target}: China's restriction on exporting rare earth metals
				\item \textbf{Demand}: Force the Chinese to halt their export restrictions of rare earths
				\item \textbf{Response}: New China News Agency calls the lawsuit "rash and unfair"; Li Weimin, Foreign Ministry spokesman said  “China hopes other countries can shoulder responsibility for supplies and can find alternative resources.”
				\item \textbf{Source}: \url{https://www.latimes.com/business/la-xpm-2012-mar-14-la-fi-obama-china-20120314-story.html}; \url{https://www.cfr.org/interview/challenging-chinas-trade-practices}; \url{https://www.cfr.org/timeline/us-china-relations}
				\end{itemize}
			\item Obama complaints at WTO on China's subsidies to aluminum producers
				\begin{itemize}
				\item	\textbf{Time}: 12 Jan 2017
				\item \textbf{Target}: China's subsidies to primary aluminum producers
				\item \textbf{Demand}: United States urges China to reduce primary aluminum excess capacity, takes action to level the playing field
				\item \textbf{Source}: \url{https://ustr.gov/about-us/policy-offices/press-office/press-releases/2017/january/Obama-Administration-Files-WTO-Complaint-China-Aluminum}
				\end{itemize}
			\item More Obama Administration's trade enforcement record on \url{chrome-extension://efaidnbmnnnibpcajpcglclefindmkaj/https://ustr.gov/sites/default/files/12152016-FACTSHEET_trade_enforcement_highlights.pdf}
			\end{itemize}
		
		\subsubsection{Trump Administration}
			\begin{itemize}
			\item Fines against ZTE for flouting US sanctions against Iran
				\begin{itemize}
				\item \textbf{Time: }7 Mar 2017
				\item \textbf{Act: }Chinese telecom giant ZTE has been fined \$1.1bn and will plead guilty to charges that it violated US rules by shipping US-made equipment to Iran and North Korea.
				\item \textbf{Response: } on 8 June 2018 ZTE agreed to pay the \$1bn fine and reshuffle its leadership. It placed an additional \$400mn in a holding account against further violations in exchange for the lift of ban. The 5-year probation period ends on 23 Mar 2022.
				\item	\textbf{Source: }\url{https://www.bbc.com/news/business-39197677}; \url{https://www.reuters.com/article/us-usa-trade-china-zte-idUSKCN1J40PR}; \url{https://www.bbc.com/news/business-44825878}; \url{https://www.globaltimes.cn/page/202203/1256649.shtml}
				\end{itemize}

			\item Tariff on washing machines and solar panels
				\begin{itemize}
				\item \textbf{Time: } 22 Jan 2018
				\item \textbf{Act: } Trump imposes 30\% tariffs on all imported washing machines and solar panels.
				\item \textbf{Response: }Beijing lodged a complaint to the WTO to help determine the legality of the US policies.
				\item \textbf{Source:} \url{https://www.reuters.com/article/us-usa-trade-china-timeline-idUSKCN1UZ24U}; \url{https://www.supplychaindive.com/news/tariff-tracker-trade-war/528842/}; \url{https://www.scmp.com/news/china/diplomacy-defence/article/2159787/china-says-us-tariffs-solar-panels-violate-trade-rules}
				\end{itemize}

			\item	Tariff on China imports following 301 investigation
				\begin{itemize}
				\item \textbf{Time}: 22 Mar 2018
				\item \textbf{Act:} Donald Trump set in motion tariffs on as much as \$60 billion in Chinese imports to the U.S. on Thursday and accused the Chinese of high-tech thievery. The list of subjects will cover aeronautics, modern rail, new-energy vehicles and high-tech products.
				\item \textbf{Response}: China responded early Friday by announcing a list of U.S. goods, including pork, apples and steel pipe, it said may be hit with higher import duties. On 2 Apr China imposes tariffs of up to 25\% on 128 US products.
				\item	\textbf{Source:} \url{https://apnews.com/article/73e5e5aa7be2408892e9904d642d2137}; \url{https://ustr.gov/about-us/policy-offices/press-office/fact-sheets/2018/march/section-301-fact-sheet}; \url{https://www.cnbc.com/2018/03/22/trump-moves-to-slap-china-with-50-billion-in-tariffs-over-intellectual-property-theft.html}
				\end{itemize}

			\item	Tariff on cars, hard disks and aircraft parts
				\begin{itemize}
				\item \textbf{Time}: 06 Jul 2018
				\item \textbf{Act:} US places 25\% duties on around US\$34 billion of imports from China, including cars, hard disks and aircraft parts. It says 25\% tariffs will also kick in on an additional \$16 billion of goods after a public comment period. 
				\item \textbf{Response}: China retaliates by imposing a 25 per cent tariff on 545 goods originating from the US worth US\$34 billion, including agricultural products, automobiles and aquatic products.
				\item	\textbf{Source:} \url{https://www.scmp.com/economy/china-economy/article/3146489/us-china-trade-war-timeline-key-dates-and-events-july-2018}; \url{https://www.cnbc.com/2018/03/22/trump-moves-to-slap-china-with-50-billion-in-tariffs-over-intellectual-property-theft.html}
				\end{itemize}
				
			\item Tariff
				\begin{itemize}
				\item \textbf{Time: } Aug 2018
				\item \textbf{Act: }Trump orders USTR to increase the tariffs on \$200 billion of Chinese imports to 25\% from the originally proposed 10\% on 1 Aug. The 10\% rate comes into effect on 24 Sep and 25\% on 1 Jan 2019, but the latter was postponed several times. The United States releases the list of \$16 billion of Chinese goods to be subject to 25\% tariffs, which comes into effect on 23 Aug. 
				\item \textbf{Response: }China retaliates with 25\% duties on \$16 billion of U.S. goods on 23 Aug, and placing customs duties on \$60bn worth of US goods on 24 Sep.
				\item	\textbf{Source: }\url{https://www.reuters.com/article/us-usa-trade-china-timeline-idUSKCN1UZ24U}; \url{https://www.scmp.com/economy/china-economy/article/3146489/us-china-trade-war-timeline-key-dates-and-events-july-2018}
				\end{itemize}
			
%			\item	Pence Speech Signals Hard-Line Approach
%				\begin{itemize}
%				\item \textbf{Time}: 04 Oct 2018
%				\item \textbf{Act:} Pence says the United States will prioritize competition over cooperation by using tariffs to combat “economic aggression.” He also condemns what he calls growing Chinese military aggression, especially in the South China Sea, criticizes increased censorship and religious persecution by the Chinese government, and accuses China of stealing American intellectual property and interfering in U.S. elections.
%				\item \textbf{Response}: Hua Chunying Makes Clear China's Position in Response to US Leader's Groundless Accusations Against China
%				\item	\textbf{Source:} \url{https://www.cfr.org/timeline/us-china-relations}; \url{https://trumpwhitehouse.archives.gov/briefings-statements/remarks-vice-president-pence-administrations-policy-toward-china/}; \url{https://www.fmprc.gov.cn/mfa_eng/xwfw_665399/s2510_665401/2535_665405/201810/t20181005_696937.html}
%				\end{itemize}
			
			\item U.S. Justice Department alleges Huawei and Meng violated trade sanctions against Iran and committed fraud and requests her extradition
				\begin{itemize}
				\item \textbf{Time}: 01 Dec 2018
				\item \textbf{Act:} Canada Arrests Meng Wanzhou
				\item \textbf{Response}: China detains two Canadian citizens, who officials accuse of undermining China’s national security
				\item	\textbf{Source:} \url{https://www.justice.gov/opa/pr/chinese-telecommunications-conglomerate-huawei-and-huawei-cfo-wanzhou-meng-charged-financial}; \url{http://www.xinhuanet.com/english/2020-08/21/c_139306059.htm}
				\end{itemize}
				
			\item Tariff
				\begin{itemize}
				\item \textbf{Time: }May 2019
				\item \textbf{Act: }After trade negotiations break down, US increases tariffs on US\$200 billion worth of Chinese goods, from 10 to 25 per cent on 10 May. US Department of Commerce announces the addition of Huawei to its “entity list” on 15 May.
				\item \textbf{Response: }China announces plans to establish its own “unreliable entity list” on 31 May and increases tariffs on US\$60 billion worth of US products on 1 June.
				\item	\textbf{Source: }\url{https://www.scmp.com/economy/china-economy/article/3146489/us-china-trade-war-timeline-key-dates-and-events-july-2018}
				\end{itemize}
			
			\item Tariff threats
				\begin{itemize}
				\item \textbf{Time: } 1 Aug 2019
				\item \textbf{Act: }After two days of trade talks with little progress and complaints by Trump that China has not followed through on a promise to buy more U.S. farm products, he announces 10\% tariffs on \$300 billion worth of Chinese imports, in addition to the 25\% already levied on \$250 billion worth of Chinese goods, to be delayed.
				\item \textbf{Response: } On 5 Aug China’s Commerce Ministry responds to the latest U.S. tariffs by halting purchases of U.S. agricultural products, and the Chinese currency, the yuan, weakens past the key seven per dollar level, sending equity markets sharply lower. The US Treasury says it has determined for the first time since 1994 that China is manipulating its currency, knocking the U.S. dollar sharply lower and sending gold prices to a six-year high.
				\item	\textbf{Source: }\url{https://www.reuters.com/article/us-usa-trade-china-timeline-idUSKCN1UZ24U}
				\end{itemize}
			
			\item Hong Kong Human Rights and Democracy Act S.1838
				\begin{itemize}
				\item \textbf{Time: }27 Nov 2019
				\item \textbf{Act: } 6 HK officials are sanctioned, including Lam Cheng Yuet-ngor and Lee Ka-chiu.
				\item \textbf{Response: } China sanctions US officials and congressmen, and human right organization officials.
				\item	\textbf{Source: }\url{https://www.bbc.com/zhongwen/trad/chinese-news-53738020}
				\end{itemize}
				
			\item Phase One Agreement
				\begin{itemize}
				\item \textbf{Time: }January 2020
				\item \textbf{Act: } China halves additional tariffs on US\$75 billion worth of American products imposed in 2019 on 14 Feb 2020, and announces a second batch of trade-war-tariff exemptions covering 79 American products on 12 May. China allows imports of barley and blueberries from the US on 14 May. Dozens of US imports from China are granted short extensions to previous tariff exemptions.
				\item \textbf{Response: } China promised to import an additional \$200 billion over the following two years and to reform its trade practices. But the more contentious structural issues regarding protection of intellectual property rights and the unfair practices by state enterprises were left for future negotiations.
				\item	\textbf{Source: }\url{https://www.prcleader.org/post/china-s-response-to-the-u-s-trade-war}; \url{https://www.scmp.com/economy/china-economy/article/3146489/us-china-trade-war-timeline-key-dates-and-events-july-2018}
				\end{itemize}
			\end{itemize}
			
		\subsubsection{Biden Administration}
			\begin{itemize}
			\item US bans Huawei, ZTE new equipment sales
				\begin{itemize}
				\item \textbf{Time: }1 Dec 2022
				\item \textbf{Act: }Biden administration has banned approvals of new telecommunications equipment from China's Huawei Technologies and ZTE because they pose "an unacceptable risk" to U.S. national security. This also effectively bar the sale or import of new equipment made by Chinese surveillance equipment maker Dahua Technology Co, video surveillance firm Hangzhou Hikvision Digital Technology Co Ltd and telecoms firm Hytera Communications Corp Ltd.
%				\item \textbf{Response: }
				\item	\textbf{Source: }\url{https://www.reuters.com/business/media-telecom/us-fcc-bans-equipment-sales-imports-zte-huawei-over-national-security-risk-2022-11-25/}
				\end{itemize}

			\item	 Uyghur Forced Labor Prevention Act H.R.6256 of 117th congress
				\begin{itemize}
				\item \textbf{Time: }12 Dec 202, come into force in 21 Jun 2022
				\item \textbf{Act: } detailed in next
				\item \textbf{Response: } detailed in next
				\item	\textbf{Source: } \url{https://www.congress.gov/bill/117th-congress/house-bill/6256?q=\%7B\%22search\%22\%3A\%22H.R.6256\%22\%7D\&s=3\&r=1}; \url{https://www.voacantonese.com/a/how-are-companies-dealing-with-uyghur-forced-labor-prevention-act-20220806/6694084.html}
				\end{itemize}
				
			\item Uyghur Forced Labor Prevention Act Strategy by DHS
				\begin{itemize}
				\item \textbf{Time: } 17 Jun 2022
				\item \textbf{Act: } No good can be imported into the US that are produced in China’s Xinjiang region or by certain entities identified in the UFLPA Strategy, unless the importer can prove by clear and convincing evidence that the goods were not produced with forced labor. The targeted goods with high prioricty are clothes, cotton, tomato and polysilicon. The advisory can be found in \url{https://www.state.gov/xinjiang-supply-chain-business-advisory/} and entity list in \url{https://www.federalregister.gov/documents/2023/06/12/2023-12481/notice-regarding-the-uyghur-forced-labor-prevention-act-entity-list}. Goods worthy of \$500 mn has been blocked by the U.S. as of 16 Mar 2023, starting from 2 Dec 2020 when the Act was first put to discussion.
				\item \textbf{Response: }China firmly opposes the Act. Cotton may still come into transnational supply chain through third country manufacturers. H\&M was detained in China in 2021.
				\item	\textbf{Source: } \url{https://www.dhs.gov/news/2022/06/17/dhs-releases-uyghur-forced-labor-prevention-act-strategy}; \url{https://www.globaltimes.cn/page/202112/1243292.shtml}; \url{https://www.shu.ac.uk/helena-kennedy-centre-international-justice/research-and-projects/all-projects/laundered-cotton}; \url{https://www.aninews.in/news/world/us/us-blocked-imports-worth-nearly-usd-500-million-due-to-uyghur-forced-labour-report20230316221505/}
				\end{itemize}

			\end{itemize}
			
	\subsection{European Countries}
%		\subsubsection{Pre-Trump Administration}
%			\begin{itemize}
%			\item	
%			\end{itemize}
		\subsubsection{Trump Administration}
			\begin{itemize}
			\item	Tariff on steel (25\%) and aluminium (10\%) products
				\begin{itemize}
				\item \textbf{Time: }1 Jun 2018
				\item \textbf{Act: } EU was imposed tariffs on steel and aluminium, affecting \$6.4 billion euro worth of EU goods.
				\item \textbf{Response: } on 22 Jun, EU retaliate with proportionate measures, affecting 2.8 billion euro worth of US imports.
				\item	\textbf{Source: } \url{https://www.intereconomics.eu/contents/year/2018/number/5/article/the-eu-response-to-us-trade-tariffs.html}
				\end{itemize}
			\end{itemize}
			
		\subsubsection{Biden Administration}
			\begin{itemize}
			\item Inflation Reduction Act
				\begin{itemize}
				\item \textbf{Time: } 16 Aug 2022
				\item \textbf{Act: } The Act provides \$500 billion to boost clean energy, reduce healthcare costs, increase tax revenues and rebuild America's economic competitiveness in the form of tax reduction, subsidy, and federal funding to green and home-grown products, especially electic-vehicles. For example, electric car buyers are eligible for a tax credit of up to \$7,500 as long as the vehicle runs on a battery built in North America with minerals mined or recycled on the continent.
				\item \textbf{Response: } EU presents the Green Deal Industrial Plan (Net-Zero Industry Act and Critical Raw Materials Act) in Feb 2023, REPowerEU in May 2022. EU member countries relax State aid rules under the Temporary Crisis Framework in Mar 2022 and its transformation Temporary Crisis and Transition Framework (TCTF) on 9 Mar 2023 in response to the IRA.
				\item	\textbf{Source: }\url{https://www.mckinsey.com/industries/public-sector/our-insights/the-inflation-reduction-act-heres-whats-in-it}; \url{https://home.treasury.gov/news/featured-stories/the-inflation-reduction-act-and-us-business-investment}; \url{https://www.politico.eu/article/trade-war-europe-us-tech/}; \url{chrome-extension://efaidnbmnnnibpcajpcglclefindmkaj/https://www.europarl.europa.eu/RegData/etudes/IDAN/2023/740087/IPOL_IDA(2023)740087_EN.pdf}
				\end{itemize}
%			\item For more on \url{https://lot.dhl.com/timeline-how-the-u-s-eu-trade-dispute-took-shape/}
			\end{itemize}
			
	\subsection{India}
%		\subsubsection{Pre-Trump Era}
%			\begin{itemize}
%			\item	
%			\end{itemize}
		\subsubsection{Trump Administration}
			\begin{itemize}
			\item Tariff on steel and aluminium
				\begin{itemize}
				\item \textbf{Time: }23 March 2018
				\item \textbf{Act: }Trump imposed tariffs of 25 percent on \$761 million of steel and of 10 percent on \$382 million of aluminium imported from India.
				\item \textbf{Response: }India proposed hiking tariffs on 30 US products in order to recoup trade penalties worth \$241 million, but never implemented.
				\item	\textbf{Source: }\url{https://www.piie.com/blogs/trade-and-investment-policy-watch/trumps-mini-trade-war-india}; \url{https://money.cnn.com/2018/06/17/news/economy/india-us-tariffs-steel-aluminum-wto}; \url{https://edition.cnn.com/2019/06/15/economy/india-tariffs-us-trump/index.html}
				\end{itemize}

			\item	Trump Ends India’s Special Trade Status
				\begin{itemize}
				\item \textbf{Time: }5 Jun 2019
				\item \textbf{Act: }The Trump administration terminates India’s preferential trade status, part of a program dating back to the 1970s that allows products from developing countries to enter the U.S. market duty free.
				\item \textbf{Response: }India announced tariffs some as high as 70\% will be imposed on 28 US products including almonds and apples from 16 June 2019.
				\item	\textbf{Source: }\url{https://www.cfr.org/timeline/us-india-relations}; \url{https://www.bbc.com/news/world-asia-india-48650505}
				\end{itemize}

			\end{itemize}
%		\subsubsection{Biden Administration}
%			\begin{itemize}
%			\item	
%			\end{itemize}
			
	\subsection{Japan}
%		\subsubsection{Pre-Trump Era}
%			\begin{itemize}
%			\item	
%			\end{itemize}
		\subsubsection{Trump Administration}
			\begin{itemize}
			\item	US-Japan trade deal
				\begin{itemize}
				\item \textbf{Time: }4 Dec 2019
				\item \textbf{Act: }The deal cutting tariffs between the countries takes effect at the beginning of next year. It cleared Japan’s upper house Wednesday after clearing the more powerful lower house earlier.
%				\item \textbf{Response: }
				\item	\textbf{Source: }\url{https://apnews.com/article/b70651046273429c9eaa7afc1a88fc9b}; \url{https://www.reuters.com/article/us-usa-trade-japan-idUSKBN1WM0A3}
				\end{itemize}
			\end{itemize}
%		\subsubsection{Biden Administration}
%			\begin{itemize}
%			\item	
%			\end{itemize}
	
		\subsection{Northern American Countries}
%		\subsubsection{Pre-Trump Era}
%			\begin{itemize}
%			\item	
%			\end{itemize}
		\subsubsection{Trump Administration}
			\begin{itemize}
			\item Tariff on steel and aluminium
				\begin{itemize}
				\item \textbf{Time: } 23 Mar 2018 Canada exempted, and extended to on 1 June.
				\item \textbf{Act: }The US tariffs hit more than  \$5.5 billion of Canadian exports of steel and nearly \$7 billion of Canadian exports of aluminum in 2017.
				\item \textbf{Response: }Canada imposed new tariffs on \$12.8 billion of US exports starting July 1. Canada's tariffs target not only American steel and aluminium but billions of dollars of US exports of agricultural and consumer products. Mexico hit back at the United States on Tuesday, imposing tariffs on around \$3 billion worth of American pork, steel, cheese and other goods in response
				\item	\textbf{Source: }\url{https://www.piie.com/blogs/trade-and-investment-policy-watch/canada-strikes-back-here-breakdown}; \url{https://www.nytimes.com/2018/06/05/us/politics/trump-trade-canada-mexico-nafta.html}
				\end{itemize}

			\item	U.S.-Mexico-Canada Agreement
				\begin{itemize}
				\item \textbf{Time: } 1 July 2020
				\item \textbf{Act: } Creating more balanced, reciprocal trade supporting high-paying jobs for Americans and grow the North American economy. Intellectual property protection is in the spotlight in this agreement. The steel and aluminium tariffs in place since Mar 2018 were lifted on 17 May 2019.
				\item	\textbf{Source: } \url{https://ustr.gov/trade-agreements/free-trade-agreements/united-states-mexico-canada-agreement}; \url{https://ustr.gov/trade-agreements/free-trade-agreements/united-states-mexico-canada-agreement/agreement-between}; \url{https://ustr.gov/usmca}; \url{https://www.supplychaindive.com/news/us-steel-tariffs-mexico-canada/555060/}
				\end{itemize}
			\end{itemize}
		\subsubsection{Biden Administration}
			\begin{itemize}
			\item	Inflation Reduction Act
				\begin{itemize}
				\item \textbf{Time: } 18 Jun 2022
				\item \textbf{Act: } above mentioned
				\item \textbf{Response: } Canada unveils an \$ 80bn investment plan in 7 Apr 2023 aimed at promoting clean energy and sustainable infrastructure in response to the US Inflation Reduction Act.
				\item	\textbf{Source: }\url{https://www.energymonitor.ai/policy/canadas-ira-response-an-80bn-clean-energy-plan/}
				\end{itemize}
			\item US and Mexico dispute on genetically modified yellow corn.
				\begin{itemize}
				\item \textbf{Time: }30 Nov 2022 starts talk, Mexico start tariff on 25 Jun 2023
%				\item \textbf{Act: }
				\item \textbf{Response: }Mexico on Saturday began imposing a 50\% tariff on white corn imports, a move the president says looks to boost national production and prevent imports of genetically modified corn.
				\item	\textbf{Source: }\url{https://apnews.com/article/science-mexico-caribbean-united-states-global-trade-be70ddbfbc17622dcc732618906feaa0}; \url{https://apnews.com/article/mexico-us-canada-gmo-corn-usmca-trade-0132baa2f950dacdde7de41a611bcb58}; \url{https://apnews.com/article/mexico-tariff-corn-canada-united-states-48b414c2dafdff8543223f88fc7bc7a7}
				\end{itemize}

			\end{itemize}
	
\section{US Policy in the Post-war Era with U.K.}
		\begin{itemize}
		\item	Smoot-Hawly Tariff Act
			\begin{itemize}
			\item \textbf{Time: }June 1930
			\item \textbf{Act: }The Smoot-Hawley Tariff Act, enacted in June 1930, added about 20\% to the United States' already high import duties on foreign agricultural products and manufactured goods. The Fordney-McCumber Act of 1922 previously raised the average import tax on foreign goods to about 40\%.
			Over the following decades, the United States steadily encouraged international trade by taking a lead role in the General Agreement on Tariffs and Trade (GATT), the North American Free Trade Agreement (NAFTA), and the World Trade Organization (WTO).
			\item \textbf{Response: }Soon, 25 countries retaliated by increasing their own tariffs. As a result, international trade declined drastically, resulting in a worldwide decline of 66\% between 1929 and 1934. Both U.S. exports and imports dropped substantially.
			\item	\textbf{Source: }\url{https://www.investopedia.com/terms/s/smoot-hawley-tariff-act.asp}
			\end{itemize}

		\end{itemize}
		

\section{US Policy with Eastern Countries in Late 20th Century}
	\subsection{Japan}
		\begin{itemize}
		\item	Semiconductor Trade Agreement
			\begin{itemize}
			\item \textbf{Time: }July 1986
			\item \textbf{Act: }the Japanese firms agreed to a suspension agreement—US trade law language for a voluntary export restraint—limiting their sales to the US market, and the Japanese government promised to address its firms’ dumping in third markets. On the third issue, the two sides agreed to a secret side letter—albeit with vaguely worded language—in which the Japanese government acknowledged the US semiconductor industry expected its sales would reach 20 percent of the Japanese market within five years
			\item	\textbf{Source: }\url{https://www.eaerweb.org/selectArticleInfo.do?article_a_no=JE0001_2020_v24n4_349\&ano=JE0001_2020_v24n4_349#T001}
			\end{itemize}
			
		\item Tariff on \$ 300mn of imports from Japan
			\begin{itemize}
			\item \textbf{Time: }Apr 1987
			\item \textbf{Act: }the Reagan administration imposed 100 percent tariffs on \$300 million of imports from Japan, including imports of major semiconductor consuming industries such as computers and televisions. In 1982, the tariff had been 4.2\% and had been mostly eliminated in Feb 1985. 
			\item \textbf{Response: }Over the rest of 1987, the Japanese government convinced its firms to cut back on output, the third-country dumping let up, and the US phased out some of the tariffs by November. The tariffs on \$165 million of imports remained in place until 1991, however, when Japan finally neared the 20 percent target, and the agreement was renegotiated and revised.
			\item	\textbf{Source: }\url{https://www.eaerweb.org/selectArticleInfo.do?article_a_no=JE0001_2020_v24n4_349\&ano=JE0001_2020_v24n4_349#T001}; \url{chrome-extension://efaidnbmnnnibpcajpcglclefindmkaj/https://www.jstor.org/stable/pdf/25063126.pdf?refreqid=excelsior\%3Aa544f1796127e4af84dad4c7dae18c25\&ab_segments=\&origin=\&initiator=\&acceptTC=1}
			\end{itemize}
		\end{itemize}
		
	\subsection{South Korea}
		\begin{itemize}
		\item	Tariff on Korean DRAM chips
			\begin{itemize}
			\item \textbf{Time: } 1992
			\item \textbf{Act: } Following Micron's antidumping petition against Samsung, Hyundai and Goldstar, the US government imposed antidumping duties against the firms. 
			\item \textbf{Response: } In 1997, the Korean government file a WTO dispute to prod for removal. In 2020 US removes duties.
			\item	\textbf{Source: }\url{https://www.eaerweb.org/selectArticleInfo.do?article_a_no=JE0001_2020_v24n4_349\&ano=JE0001_2020_v24n4_349#T001}; \cite{Bown2020}
			\end{itemize}
			
		\item	Tariff and fine on Korean DRAM chips
			\begin{itemize}
			\item \textbf{Time: } 2002-2005
			\item \textbf{Act: } US Department of Justice launches investigation that DRAM price-fixing that hurt Dell, Compaq, Hewlett-Packard, Apple, IBM, and Gateway. In 2003, Micron executive pleads guilty to obstruction of justice violation. Infineon (2004, \$160 million); Hynix (2004, \$185 million); and Samsung (2005, \$300 million) also plead guilty and pay fines. 
			\item \textbf{Response: } South Korea files separate WTO dispute against 2002 US and EU duties on DRAMS in 2003. Dispute resolved in 2008, when United States and European Union remove duties.
			\item	\textbf{Source: } \cite{Bown2020}
			\end{itemize}
		\end{itemize}
		
	\subsection{Hong Kong}
		\begin{itemize}
		\item	Uruguay Round under the GATT in late 1980s
		\item the US-Hong Kong Policy Act of 1992
				\begin{itemize}
				\item \textbf{Time: }1992
				\item \textbf{Act: }The Act also indicates that, after June 30, 1997, it is the sense of Congress that “the United States should seek to maintain and expand economic and trade relations with Hong Kong and should continue to treat Hong Kong as a separate territory”. 
				\item \textbf{Response: }
				\item	\textbf{Source: }\url{https://www.asiaglobalonline.hku.hk/what-next-hong-kong-united-states-trade}; \url{https://www.congress.gov/bill/102nd-congress/senate-bill/1731/text}
				\end{itemize}
		\end{itemize}
		
	\subsection{Taiwan}
		\begin{itemize}
		\item	Tariff on Taiwanese RAM chips
			\begin{itemize}
			\item \textbf{Time: } 1997
			\item \textbf{Act: } Following Micron's antidumping petition, the US government imposed antidumping duties. Duties on Taiwan include a Texas Instruments joint venture with Acer.
			\item	\textbf{Source: }\cite{Bown2020}
			\end{itemize}
		\end{itemize}
		
		
\section{Paper Summary}
	\subsection{Innovation and Trade Policy}
	
	\subsection{Measurement of Innovation/ Tech Licensing}
	
	\subsection{Date Sources}
	
	Akcigit-ATes-Impullitti (2018) Innovation and Trade Policy in a Globalized World \cite{AkcigitAtesImpullitti2018}
 
Ana Maria Santacreu’s papers (check her website):  
Dynamic Gains from Trade Agreements with Intellectual Property Provisions \cite{Santacreu2022}
International Technology Licensing, Intellectual Property Rights and Tax Havens \cite{Santacreu2023}
Knowledge Diffusion, Trade and Innovation across Countries and Sectors  \cite{CaiLiSantacreu2022}
 Innovation, Diffusion, and Trade: Theory and Measurement.  \cite{Santacreu2015}

Linyi Cao, Helu Jiang, Guangwei Li, and Lijun Zhu. Haste makes waste? quantity-based subsidies under heterogeneous innovations. Journal of Monetary Economics, Forthcoming.\cite{CaoJiangLiZhu2023}





\newpage

\footnotesize
\bibliographystyle{apalike}
\bibliography{Tech_war}

Others references:
\begin{itemize}
	\item The U.S.-China Trade War and the Tariff Weapons \url{https://www.wilsoncenter.org/publication/us-china-trade-war-and-tariff-weapons}
	\item United States International Trade Commission \url{https://www.usitc.gov/}
	\item Fajgelbaum, and Khandelwal, The Economic Impacts of the US-China Trade War, 2021, NBER \url{chrome-extension://efaidnbmnnnibpcajpcglclefindmkaj/https://www.nber.org/system/files/working_papers/w29315/w29315.pdf}
	\item Trump's Trade War Timeline \url{https://www.piie.com/blogs/trade-and-investment-policy-watch/trumps-trade-war-timeline-date-guide}
	\item EU-US trade relation, w.r.t. the EU-US Trade and Technology Council from 15 June 2021 \url{https://policy.trade.ec.europa.eu/eu-trade-relationships-country-and-region/countries-and-regions/united-states_en}
\end{itemize}

\end{document}